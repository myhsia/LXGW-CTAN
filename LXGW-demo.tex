\documentclass{l3doc}

\usepackage[mono = false, osf]{libertine}
\tracinglostchars = 0
\usepackage[quiet, LoadFandol = false, PunctStyle = plain]{xeCJK}
\defaultfontfeatures{Path = ../unpacked/, Scale = .875}
\usepackage{booktabs, graphicx, tikz}
\ExplSyntaxOn \makeatletter
\DeclareDocumentCommand \key { s m }
  {
    \IfBooleanTF {#1} { \textcolor{red}{#2} }
      {
        \seq_set_from_clist:Nn \l_tmpa_seq {#2}
        \seq_set_map:NNn \l_tmpb_seq \l_tmpa_seq
          { \exp_not:n { \textcolor{red}{##1} } }
        \seq_use:Nn \l_tmpb_seq { ,~ } \,\textbar\,
      }
  }
\DeclareCommandCopy \val \meta
\newlist{keyval}{itemize}{10}
\setlist[keyval]{leftmargin = 0pt, labelsep = 0pt}
\def \HoLogo@LXGW#1{\xfontspec{LXGWWenKaiLite-Medium}LX\kern-.05emG\kern-.05emW}
\DeclareDocumentCommand \xfontspec { O{} m O{} }
  {
    \fontspec    {#2} [#1, #3] \selectfont
    \CJKfontspec {#2} [#1, #3] \selectfont
  }
\makeatother \ExplSyntaxOff

\begin{document}

\makeatletter
\title{The \hologo{LXGW} Font Family\thanks{%
  \url{https://github.com/lxgw},
  \url{https://github.com/myhsia/LXGW-CTAN}
}\texorpdfstring{\enspace \textbar \enspace\parbox{4em}{%
    \xfontspec{LXGWWenKaiLite-Regular}%
    \scalebox{\fpeval{4/7}}{落霞与孤鹜齐飞}\\[-.75em]
    \scalebox{\fpeval{4/7}}{秋水共长天一色}%
  }}{}%
}

\author{%
  Maintainer: \hologo{LXGW}\thanks{\texttt{%
    \href{mailto:calxgw2018@gmail.com}{calxgw2018@gmail.com}}},
  Administrator: Mingyu Xia\thanks{\texttt{%
    \href{mailto:xiamingyu@westlake.edu.cn}{xiamingyu@westlake.edu.cn}%
  }}
}

\date{Released 2025-12-01\quad \texttt{v1.521C}}

\def\Copyright{%
  \begin{center}
    \fbox{\fbox{\parbox {.8\linewidth} {%
    This package packed a selection of open-source fonts from the
    \href{https://github.com/lxgw/LxgwWenKai-Lite}
      {\xfontspec{LXGWWenKaiLite-Regular}%
        LXGW WenKai\textsuperscript{Lite Edition}},
    \href{https://github.com/lxgw/LxgwWenKaiGB-Lite}
      {\xfontspec{LXGWWenKaiGBLite-Regular}%
        LXGW WenKai GB\textsuperscript{Lite Edition}},
    \href{https://github.com/lxgw/LxgwMarkerGothic}
      {\xfontspec{LXGWMarkerGothic-Regular}%
        LXGW Marker Gothic},
    \href{https://github.com/lxgw/kose-font}
      {\xfontspec{LXGWXiaolai-Regular}%
        Xiaolai Font}, and
    \href{https://github.com/lxgw/yozai-font}
      {\xfontspec{LXGWYozai-Regular}%
        Yozai Font},
    which are released into public domain by
    \href{https://lxgw.github.io/}{\hologo{LXGW}} (2021).
    They are licensed under the
    \href{https://ctan.org/license/ofl}{SIL Open Font License (OFL)}.
  }}}
  \end{center}}
\def\@thanks{\Copyright}

\maketitle
\tikz[remember picture, overlay]
  \node [ rotate = -10, font = \xfontspec{LXGWWenKaiLite-Regular},
          scale = 50, opacity = .1 ] at
        ([shift = {(10 * \f@size \p@, 8 * \f@size \p@)}]
          current page.south west) {鹜};
\makeatother

\begin{abstract}
  The \hologo{LXGW} Font Family provides an unprofessional open-source Chinese
  font family, which (will be) included in the \pkg{ctex-kit} as a |fontset|
  option.
\end{abstract}

\noindent
The following lists the Chinese name, English name, filename, and demo of the
fonts\footnote{%
  To enhance visual effects, \cs{makebox} with |s| parameter is applied
  to the two lines of every demo to make them the same width, and the kernings
  between characters make no sense here; Punctuation compression is disabled
  in \pkg{xeCJK}.}:
Cantonese, English, Chinese (Simplified / Traditional), and
Japanese versions of ``\textbf{I Can Eat Glass}'', arabic numbers, and Missing
character markers are provided.
\def\0{%
  \makebox[\linewidth][s]{%
  我可以食玻璃,\textbf{佢傷唔到我㗎。}%
  I can eat glass \textbf{and it doesn't hurt me.}%
  我能吞下玻璃\textbf{而不伤身体。}}\\
  \makebox[\linewidth][s]{我能吞下玻璃而不傷身體。%
  私はガラスを食べられます。\textbf{それは私を傷つけません。}%
  12345\textbf{67890} ^^^^2370}
}
\begin{keyval}
  \item [\key{\xfontspec[BoldFont = LXGWWenKaiLite-Medium]%
          {LXGWWenKaiLite-Regular}霞鹜文楷 (LXGW WenKai)}]
        \file{LXGWWenKaiLite-Regular.ttf}, \file{LXGWWenKaiLite-Medium.ttf}
  \begin{center}
    \xfontspec[BoldFont = LXGWWenKaiLite-Medium]%
      {LXGWWenKaiLite-Regular}\0
  \end{center}
  \item [\key{\xfontspec[BoldFont = LXGWWenKaiGBLite-Medium]%
          {LXGWWenKaiGBLite-Regular}%
          霞鹜文楷\textsubscript{\textbf{国标}}
          (LXGW WenKai\textsubscript{\textbf{GB}})}]
        \file{LXGWWenKaiGBLite-Regular.ttf}, \file{LXGWWenKaiGBLite-Medium.ttf}
  \begin{center}
    \xfontspec[BoldFont = LXGWWenKaiGBLite-Medium]%
      {LXGWWenKaiGBLite-Regular}\0
  \end{center}
  \item [\key{\xfontspec[AutoFakeBold]{LXGWMarkerGothic-Regular}%
          霞鹜漫黑 (LXGW Marker Gothic)}]
        \file{LXGWMarkerGothic-Regular.ttf}
  \begin{center}
    \xfontspec[AutoFakeBold]{LXGWMarkerGothic-Regular}\0
  \end{center}
  \item [\key{\xfontspec[AutoFakeBold]{LXGWXiaolai-Regular}%
          小赖字体 (Xiaolai Font)}]
        \file{LXGWXiaolai-Regular.ttf}
  \begin{center}
    \xfontspec[AutoFakeBold]{LXGWXiaolai-Regular}\0
  \end{center}
  \item [\key{\xfontspec[AutoFakeBold]{LXGWYozai-Regular}%
          悠哉字体 (Yozai Font)}]
        \file{LXGWYozai-Regular.ttf}, \file{LXGWYozai-Medium.ttf}
  \begin{center}
    \xfontspec[AutoFakeBold]{LXGWYozai-Regular}\0
  \end{center}
\end{keyval}

\end{document}