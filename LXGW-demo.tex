\documentclass{l3doc}

\usepackage[mono = false, osf]{libertine}
\usepackage[fontset = lxgw, scheme = plain]{ctex}
\tracinglostchars = 0
\XeTeXcharclass "2370 = 1
\defaultfontfeatures{Extension = .ttf}
\usepackage{booktabs, graphicx, tikz}
\ExplSyntaxOn \makeatletter

\DeclareDocumentCommand \key { s m }
  {
    \IfBooleanTF {#1} { \textcolor{red}{#2} }
      {
        \seq_set_from_clist:Nn \l_tmpa_seq {#2}
        \seq_set_map:NNn \l_tmpb_seq \l_tmpa_seq
          { \exp_not:n { \textcolor{red}{##1} } }
        \seq_use:Nn \l_tmpb_seq { ,~ } \,\textbar\,
      }
  }
\DeclareCommandCopy \val \meta
\newlist{keyval}{itemize}{10}
\setlist[keyval]{leftmargin = 0pt, labelsep = 0pt}
\def \HoLogo@LXGW#1{%
  \fontspec{LXGWWenKaiLite-Medium}\selectfont
  LX\kern-.05emG\kern-.05emW}
\makeatother \ExplSyntaxOff

\begin{document}

\makeatletter
\title{The \hologo{LXGW} Font Family\thanks{%
  \url{https://github.com/lxgw},
  \url{https://github.com/myhsia/LXGW-CTAN}
}\texorpdfstring{\enspace \textbar \enspace\parbox{4em}{%
    \songti
    \scalebox{\fpeval{4/7}}{落霞与孤鹜齐飞}\\[-.75em]
    \scalebox{\fpeval{4/7}}{秋水共长天一色}%
  }}{}%
}

\author{%
  Maintainer: \hologo{LXGW}\thanks{\texttt{%
    \href{mailto:calxgw2018@gmail.com}{calxgw2018@gmail.com}}},
  Administrator: Mingyu Xia\thanks{\texttt{%
    \href{mailto:xiamingyu@westlake.edu.cn}{xiamingyu@westlake.edu.cn}%
  }}
}

\date{Released 2025-12-03\quad \texttt{v1.521D}}

\def\Copyright{%
  \begin{center}
    \fbox{\fbox{\parbox {.8\linewidth} {%
    This package packed a selection of open-source fonts from the
    \href{https://github.com/lxgw/LxgwWenKai-Lite}
      {\songti 霞鹜文楷\textsuperscript{轻便版}},
    \href{https://github.com/lxgw/LxgwWenKaiGB-Lite}
      {\CJKfontspec{LXGWWenKaiGBLite-Regular}\selectfont
        霞鹜文楷\ 国标\textsuperscript{轻便版}},
    \href{https://github.com/lxgw/LxgwMarkerGothic}
      {\heiti 霞鹜漫黑},
    \href{https://github.com/lxgw/kose-font}
      {\fangsong 小赖字体}, and
    \href{https://github.com/lxgw/yozai-font}
      {\kaishu 悠哉字体},
    which are released into public domain by
    \href{https://lxgw.github.io/}{\hologo{LXGW}} (2021).
    They are licensed under the
    \href{https://ctan.org/license/ofl}{SIL Open Font License (OFL)}.
  }}}
  \end{center}}
\def\@thanks{\Copyright}

\maketitle
\tikz[remember picture, overlay]
  \node [ rotate = -10, font = \songti,
          scale = 50, opacity = .1 ] at
        ([shift = {(10 * \f@size \p@, 8 * \f@size \p@)}]
          current page.south west) {鹜};
\makeatother

\begin{abstract}
  The \hologo{LXGW} Font Family provides an unprofessional open-source Chinese
  font family, which (will be) included in the \pkg{ctex-kit} as a |fontset|
  option.
\end{abstract}

\noindent
The following lists the Chinese name, English name, filename, and demo of the
fonts\footnote{%
  To enhance visual effects, \cs{makebox} with |s| parameter is applied
  to the two lines of every demo to make them the same width, and the kernings
  between characters make no sense here; Punctuation compression is disabled
  in \pkg{xeCJK}.}:
Cantonese, Japanese, Chinese (Simplified / Traditional)
versions of ``\textbf{I Can Eat Glass}'', and Missing
character markers are provided.
\def\0{%
  \makebox[\linewidth][s]{%
    我可以食玻璃,\textbf{佢傷唔到我㗎。}%
    私はガラスを食べられます。\textbf{それは私を傷つけません。}}\\
  \makebox[\linewidth][s]{%
    我能吞下玻璃而不伤身体。\textbf{我能吞下玻璃而不伤身体。}%
    我能吞下玻璃而不傷身體。^^^^2370\textbf{^^^^2370}
  }
}
\begin{keyval}
  \item[\key{\songti 霞鹜文楷 (\pkg{LXGW WenKai})}]
        \file{LXGWWenKaiLite-Regular}, \file{LXGWWenKaiLite-Medium}
  \begin{center}
    \songti\0
  \end{center}
  \item[\key{\CJKfontspec[BoldFont = LXGWWenKaiGBLite-Medium]%
               {LXGWWenKaiGBLite-Regular}\selectfont
        霞鹜文楷\textsubscript{\textbf{国标}}
        (\pkg{LXGW WenKai\textsubscript{\textbf{GB}}})}]
        \file{LXGWWenKaiGBLite-Regular.ttf}, \file{LXGWWenKaiGBLite-Medium.ttf}
  \begin{center}
    \CJKfontspec[BoldFont = LXGWWenKaiGBLite-Medium]%
      {LXGWWenKaiGBLite-Regular}\selectfont\0
  \end{center}
  \item[\key{\heiti 霞鹜漫黑 (\pkg{LXGW Marker Gothic})}]
        \file{LXGWMarkerGothic-Regular.ttf}
  \begin{center}
    \heiti\0
  \end{center}
  \item[\key{\fangsong 小赖字体 (\pkg{Xiaolai Font})}]
        \file{LXGWXiaolai-Regular.ttf}
  \begin{center}
    \fangsong\0
  \end{center}
  \item[\key{\kaishu 悠哉字体 (\pkg{Yozai Font})}]
        \file{LXGWYozai-Regular.ttf}, \file{LXGWYozai-Medium.ttf}
  \begin{center}
    \kaishu\0
  \end{center}
\end{keyval}

\end{document}