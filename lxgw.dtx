% \iffalse meta-comment
%
% File: lxgw.dtx
% -----------------------------------------------------------------------
%   Runtime: Copyright (C) 2025 by Mingyu Xia <myhsia@outlook.com>      *
%                                                                       *
%   Fonts:   Copyright (C) 2021-2025 by LXGW  <https://github.com/lxgw> *
%                                                                       *
%   This work may be distributed and/or modified under the conditions   *
%   of the SIL Open Font License (OFL) version 1.1.                     *
%   The latest version of this license is in                            *
%                                                                       *
%       https://openfontlicense.org                                     *
%                                                                       *
%   This work has the OFL maintenance status `maintained'.              *
%                                                                       *
%   The Current Maintainer of this work on CTAN is Mingyu Xia.          *
%                                                                       *
%   The Current Maintainer of this font family is LXGW.                 *
%                                                                       *
%   This work consists of the files lxgw.dtx,                           *
%                                   lxgw.ins,                           *
%                 the derived files ctex-fontset-lxgw.def,              *
%                                   ctex-zhmap-lxgw.tex,                *
%                    the font files LXGWWenKaiLite-Regular.ttf,         *
%                                   LXGWWenKaiLite-Medium.ttf,          *
%                                   LXGWWenKaiGBLite-Regular.ttf,       *
%                                   LXGWWenKaiGBLite-Medium.ttf,        *
%                                   LXGWMarkerGothic-Regular.ttf,       *
%                                   LXGWXiaolai-Regular.ttf,            *
%                                   LXGWYozai-Regular.ttf,              *
%                                   LXGWYozai-Medium.ttf,               *
%           the documentation files lxgw.pdf,                           *
%                               and README.md.                          *
% -----------------------------------------------------------------------
%
%   Any modification of this file should ensure that the copyright and
%   license information is placed in the derived files.
%
% -----------------------------------------------------------------------
%
%<*internal>
\iffalse
%</internal>
%
%<*readme>
[![CTAN Version](https://img.shields.io/ctan/v/lxgw-fonts)](https://ctan.org/pkg/lxgw-fonts)
[![GitHub Last Commit](https://img.shields.io/github/last-commit/myhsia/LXGW-CTAN)](https://github.com/myhsia/LXGW-CTAN/commits)
[![GitHub Repo stars](https://img.shields.io/github/stars/myhsia/LXGW-CTAN)](https://github.com/myhsia/LXGW-CTAN)

The `LXGW` Font Family
======================

The `LXGW` Font Family provides an unprofessional open-source Chinese font family.

Issues
------

The issue tracker for `LXGW` is currently located
[on GitHub](https://github.com/myhsia/LXGW-CTAN).

Copyright and License
---------------------

Runtime: Copyright (C) 2025 by Mingyu Xia <myhsia@outlook.com>

Fonts:   Copyright (C) 2021-2025 by LXGW  <https://github.com/lxgw>
  
This work may be distributed and/or modified under the conditions
of the SIL Open Font License (OFL) version 1.1.
The latest version of this license is in

    https://openfontlicense.org

This work has the OFL maintenance status `maintained'.

The Current Maintainer of this work on CTAN is Mingyu Xia.

The Current Maintainer of this font family is LXGW.
%</readme>
%
%<*internal>
\fi
%</internal>
%
%<*driver>
\documentclass{l3doc}
\usepackage[mono = false, osf]{libertine}
\usepackage[punct = plain, fontset = lxgw, scheme = plain]{ctex}
\defaultfontfeatures{Extension = .ttf, Scale = .95}
\usepackage{tikz}
\ExplSyntaxOn \makeatletter
\tex_tracinglostchars:D = 0
\tex_XeTeXcharclass:D "2370 = 1
\DeclareDocumentCommand \key { s m }
  {
    \IfBooleanTF {#1} { \color_select:n { red } #2 ~ }
      {
        \seq_set_from_clist:Nn \l_tmpa_seq {#2}
        \seq_set_map:NNn \l_tmpb_seq \l_tmpa_seq
          {
            \color_group_begin:
            \color_select:n { red } \exp_not:n { ##1 }
            \color_group_end:
          }
        \seq_use:Nn \l_tmpb_seq { ,~ } \:=\:
      }
  }
\DeclareCommandCopy \val \meta
\newlist{keyval}{itemize}{10}
\setlist[keyval]{leftmargin = 0pt, labelsep = 0pt}
\def \HoLogo@LXGW#1{^^A
  \tex_XeTeXcharclass:D "004C = 1^^A L
  \tex_XeTeXcharclass:D "0058 = 1^^A X
  \tex_XeTeXcharclass:D "0047 = 1^^A G
  \tex_XeTeXcharclass:D "0057 = 1^^A W
  LX\kern-.05emG\kern-.05emW}
\makeatother \ExplSyntaxOff

\begin{document}
  \DocInput{\jobname.dtx}
\end{document}
%</driver>
% \fi
%
% \makeatletter
% \title{The \hologo{LXGW} Font Family\thanks{^^A
%   \url{https://github.com/lxgw},
%   \url{https://github.com/myhsia/LXGW-CTAN}
% }\texorpdfstring{\enspace \textbar \enspace\parbox{4em}{^^A
%     \songti
%     \scalebox{\fpeval{4/7}}{落霞与孤鹜齐飞}\\[-.75em]
%     \scalebox{\fpeval{4/7}}{秋水共长天一色}^^A
%   }}{}^^A
% }
%
% \author{^^A
%   Maintainer: \hologo{LXGW} (落霞孤鹜),
%   Administrator: Mingyu Xia (夏明宇)\thanks{\texttt{^^A
%     \href{mailto:xiamingyu@westlake.edu.cn}{xiamingyu@westlake.edu.cn}^^A
%   }}
% }
%
% \date{Released 2025-12-05\quad \texttt{v1.521F}}
%
% \def\Copyright{^^A
%   \begin{center}
%   \setlength \fboxrule {1.2pt} \fbox{\setlength \fboxrule {.6pt}^^A
%     \fbox{\parbox {.8\linewidth} {^^A
%     This package packs a selection of open-source fonts from the
%     \href{https://github.com/lxgw/LxgwWenKai-Lite}
%       {\songti 霞鹜文楷\textsuperscript{轻便版}},
%     \href{https://github.com/lxgw/LxgwWenKaiGB-Lite}
%       {\CJKfontspec{LXGWWenKaiGBLite-Regular}\selectfont
%         霞鹜文楷\ 国标\textsuperscript{轻便版}},
%     \href{https://github.com/lxgw/LxgwMarkerGothic}
%       {\heiti 霞鹜漫黑},
%     \href{https://github.com/lxgw/kose-font}
%       {\fangsong 小赖字体}, and
%     \href{https://github.com/lxgw/yozai-font}
%       {\kaishu 悠哉字体},
%     which are released into public domain by
%     \href{https://lxgw.github.io/}{\hologo{LXGW}} since 2021.
%     They are licensed under the
%     \href{https://ctan.org/license/ofl}{SIL Open Font License (OFL)}.
%   }}}
%   \end{center}}
% \def\@thanks{\Copyright}
%
% \maketitle
% \tikz[remember picture, overlay]
%   \node [ rotate = -10, font = \songti,
%           scale = 50, opacity = .1 ] at
%         ([shift = {(10 * \f@size \p@, 8 * \f@size \p@)}]
%           current page.south west) {鹜};
% \makeatother
%
% \begin{documentation}
%
% \begin{abstract}
%   The \hologo{LXGW} Font Family provides an unprofessional open-source Chinese
%   font family, which (will be) included in the \pkg{ctex-kit} as a |fontset|
%   option.
% \end{abstract}
%
% \section{Font Demos}
%
% The following lists the Chinese name, English name, filename, and demos of the
% fonts: Cantonese, Japanese, Chinese (Simplified / Traditional)
% versions of ``\textbf{I Can Eat Glass}'', missing
% character markers are provided with punctuation compression disabled and
% fulfilling line.
% \def\0{^^A
%   \makebox[\linewidth][s]{^^A
%     我可以食玻璃,\textbf{佢傷唔到我㗎。}^^A
%     私はガラスを食べられます。\textbf{それは私を傷つけません。}}\\
%   \makebox[\linewidth][s]{^^A
%     我能吞下玻璃而不伤身体。\textbf{我能吞下玻璃而不伤身体。}^^A
%     我能吞下玻璃而不傷身體。^^^^2370^^^^2370^^^^2370^^A
%   }^^A
% }
% \begin{keyval}
%   \item[\key*{\songti 霞鹜文楷 (\pkg{LXGW WenKai})}]
%         \file{LXGWWenKaiLite-Regular.ttf}, \file{LXGWWenKaiLite-Medium.ttf}
%   \begin{center}
%     \songti\0
%   \end{center}
%   \item[\key*{\CJKfontspec[BoldFont = LXGWWenKaiGBLite-Medium]^^A
%                {LXGWWenKaiGBLite-Regular}\selectfont
%         霞鹜文楷\textsubscript{\textbf{国标}}
%         (\pkg{LXGW WenKai\textsubscript{\textbf{GB}}})}]
%         \file{LXGWWenKaiGBLite-Regular.ttf},
%         \file{LXGWWenKaiGBLite-Medium.ttf}
%   \begin{center}
%     \CJKfontspec[BoldFont = LXGWWenKaiGBLite-Medium]^^A
%       {LXGWWenKaiGBLite-Regular}\selectfont\0
%   \end{center}
%   \item[\key*{\heiti 霞鹜漫黑 (\pkg{LXGW Marker Gothic})}]
%         \file{LXGWMarkerGothic-Regular.ttf}
%   \begin{center}
%     \heiti\0
%   \end{center}
%   \item[\key*{\fangsong 小赖字体 (\pkg{Xiaolai Font})}]
%         \file{LXGWXiaolai-Regular.ttf}
%   \begin{center}
%     \fangsong\0
%   \end{center}
%   \item[\key*{\kaishu 悠哉字体 (\pkg{Yozai Font})}]
%         \file{LXGWYozai-Regular.ttf}, \file{LXGWYozai-Medium.ttf}
%   \begin{center}
%     \kaishu\0
%   \end{center}
% \end{keyval}
%
% \end{documentation}
%
% \begin{implementation}
%
% \section{The Source Code}
%
% \subsection{The \texttt{ctex-fontset-lxgw.def} file}
%
% Start the optionlist |fontset| for l3docstrip.
%    \begin{macrocode}
%<*fontset>
%    \end{macrocode}
% Declare the \pkg{ctex-kit} font configuration file with date, version, and
% description.
%    \begin{macrocode}
\ProvidesExplFile{ctex-fontset-lxgw.def}
  {2025-12-05} {1.521F} {lxgw fontset configuration for ctex-kit}
%    \end{macrocode}
%    \begin{macrocode}
\ctex_fontset_case:nnnn
%    \end{macrocode}
%    \begin{macrocode}
  { \ctex_fontset_error:n { lxgw } }
%    \end{macrocode}
%    \begin{macrocode}
  {
    \ctex_zhmap_case:nnn
      {
        \setCJKmainfont { LXGWWenKaiLite-Regular.ttf   }
          [
            BoldFont   = LXGWWenKaiLite-Medium.ttf,
            ItalicFont = LXGWYozai-Regular.ttf,
          ]
        \setCJKsansfont { LXGWMarkerGothic-Regular.ttf } [ AutoFakeBold ]
        \setCJKmonofont { LXGWXiaolai-Regular.ttf      } [ AutoFakeBold  ]
        \setCJKfamilyfont { zhsong } { LXGWWenKaiLite-Regular.ttf   }
          [ BoldFont = LXGWWenKaiLite-Medium.ttf ]
        \setCJKfamilyfont { zhhei  } { LXGWMarkerGothic-Regular.ttf }
          [ AutoFakeBold ]
        \setCJKfamilyfont { zhfs   } { LXGWXiaolai-Regular.ttf      }
          [ AutoFakeBold ]
        \setCJKfamilyfont { zhkai  } { LXGWYozai-Regular.ttf        }
          [ BoldFont = LXGWYozai-Medium.ttf ]
%    \end{macrocode}
%    \begin{macrocode}
        \ctex_punct_set:n { lxgw }
        \ctex_punct_map_family:nn   { \CJKrmdefault         } { zhsong  }
        \ctex_punct_map_family:nn   { \CJKsfdefault         } { zhhei   }
        \ctex_punct_map_family:nn   { \CJKttdefault         } { zhfs    }
        \ctex_punct_map_bfseries:nn { \CJKrmdefault, zhsong } { zhsongb }
        \ctex_punct_map_bfseries:nn { \CJKsfdefault, zhhei  } { zhheib  }
        \ctex_punct_map_itshape:nn  { \CJKrmdefault         } { zhkai   }
      }
%    \end{macrocode}
%    \begin{macrocode}
      {
        \ctex_load_zhmap:nnnn { rm } { zhhei } { zhfs } { lxgw }
        \ctex_punct_set:n { lxgw }
        \ctex_punct_map_family:nn   { \CJKrmdefault } { zhsong }
        \ctex_punct_map_bfseries:nn { \CJKrmdefault } { zhhei  }
        \ctex_punct_map_itshape:nn  { \CJKrmdefault } { zhkai  }
      }
%    \end{macrocode}
%    \begin{macrocode}
      { \ctex_fontset_error:n { lxgw } }
  }
%    \end{macrocode}
%    \begin{macrocode}
  {
    \ctex_set_upmap_unicode:nnn { upserif   }
      { LXGWWenKaiLite-Regular.ttf   } { LXGWWenKaiLite-Medium.ttf }
    \ctex_set_upmap_unicode:nnn { upsans    }
      { LXGWMarkerGothic-Regular.ttf } { }
    \ctex_set_upmap_unicode:nnn { upmono    }
      { LXGWXiaolai-Regular.ttf      } { }
    \ctex_set_upmap_unicode:nnn { upserifit }
      { LXGWYozai-Regular.ttf        } { }
    \ctex_set_upfamily:nnn { zhsong } { upzhserif    } { upzhserifb }
    \ctex_set_upfamily:nnn { zhhei  } { upzhsans     } { }
    \ctex_set_upfamily:nnn { zhfs   } { upzhmono     } { }
    \ctex_set_upfamily:nnn { zhkai  } { upzhserifit  } { }
  }
%    \end{macrocode}
%    \begin{macrocode}
  {
    \setCJKmainfont { LXGWWenKaiLite-Regular }
      [
        Extension  = .ttf,
        BoldFont   = LXGWWenKaiLite-Medium,
        ItalicFont = LXGWYozai-Regular
      ]
    \setCJKsansfont { LXGWMarkerGothic-Regular }
      [ Extension = .ttf, AutoFakeBold ]
    \setCJKmonofont { LXGWXiaolai-Regular      }
      [ Extension = .ttf, AutoFakeBold  ]
    \setCJKfamilyfont { zhsong } { LXGWWenKaiLite-Regular   }
      [ Extension = .ttf, BoldFont = LXGWWenKaiLite-Medium ]
    \setCJKfamilyfont { zhhei  } { LXGWMarkerGothic-Regular }
      [ Extension = .ttf, AutoFakeBold ]
    \setCJKfamilyfont { zhfs   } { LXGWXiaolai-Regular      }
      [ Extension = .ttf, AutoFakeBold ]
    \setCJKfamilyfont { zhkai  } { LXGWYozai-Regular        }
      [ Extension = .ttf, BoldFont = LXGWYozai-Medium ]
  }
%    \end{macrocode}
%    \begin{macrocode}
\NewDocumentCommand \songti   { } { \CJKfamily { zhsong } }
\NewDocumentCommand \heiti    { } { \CJKfamily { zhhei  } }
\NewDocumentCommand \fangsong { } { \CJKfamily { zhfs   } }
\NewDocumentCommand \kaishu   { } { \CJKfamily { zhkai  } }
%    \end{macrocode}
% End the optionlist |fontset| for l3docstrip.
%    \begin{macrocode}
%</fontset>
%    \end{macrocode}
%
% \subsection{The \texttt{ctex-zhmap-lxgw.tex} file}
%
% Start the optionlist |zhmap| for l3docstrip.
%    \begin{macrocode}
%<*zhmap>
%    \end{macrocode}
% Forked from the |zhmap| optionlist of \file{ctex.dtx}\footnote{^^A
%   \url{https://github.com/CTeX-org/ctex-kit/blob/master/ctex/ctex.dtx}}.
%
%    \begin{macrocode}
\begingroup\catcode61\catcode48\catcode32=10\relax%
  \catcode 35=6  % #
  \catcode 45=12 % -
  \catcode123=1  % {
  \catcode125=2  % }
  \toks0{\endlinechar=\the\endlinechar\relax}%
  \toks2{\endlinechar=-1 }%
  \def\x#1 #2 {%
    \toks0\expandafter{\the\toks0 \catcode#1=\the\catcode#1\relax}%
    \toks2\expandafter{\the\toks2 \catcode#1=#2 }}%
  \x  13  5 % carriage return
  \x  32 10 % space
  \x  35  6 % #
  \x  40 12 % (
  \x  41 12 % )
  \x  45 12 % -
  \x  46 12 % .
  \x  47 12 % /
  \x  58 12 % :
  \x  60 12 % <
  \x  61 12 % =
  \x  64 11 % @
  \x  91 12 % [
  \x  93 12 % ]
  \x 123  1 % {
  \x 125  2 % }
  \edef\x#1{\endgroup%
    \edef\noexpand#1{%
      \the\toks0 %
      \let\noexpand\noexpand\noexpand#1%
          \noexpand\noexpand\noexpand\undefined%
      \noexpand\noexpand\noexpand\endinput}%
    \the\toks2}%
\expandafter\x\csname ctex@zhmap@endinput\endcsname
\begingroup\expandafter\endgroup
\expandafter\let\csname ifzhmappdf\expandafter\endcsname\csname
  \expandafter\ifx\csname ifctexpdf\endcsname\relax
    \expandafter\ifx\csname pdfoutput\endcsname\relax
      iffalse\else\ifnum\pdfoutput < 1 iffalse\else iftrue\fi\fi
  \else ifctexpdf\fi
\endcsname
\begingroup
\expandafter\ifx\csname ProvidesFile\endcsname\relax
  \long\def\x#1\ProvidesFile#2[#3]{%
    #1%
    \immediate\write-1{File: #2 #3}%
    \expandafter\xdef\csname ver@#2\endcsname{#3}}
  \expandafter\x%
\fi
\endgroup
%    \end{macrocode}
% Provides the identification information of the font mapping file.
%    \begin{macrocode}
\ProvidesFile{ctex-zhmap-lxgw.tex}%
  [2025-12-05 v1.521F lxgw font map loader for DVIPDFMx (CTEX)]
%    \end{macrocode}
% The font mapping configurations.
%    \begin{macrocode}
\ifzhmappdf
%    \end{macrocode}
% Since \hologo{pdfTeX} maps too slowly, this mode is obsolete.
%    \begin{macrocode}
\else
  \special{pdf:mapline gbk@UGBK@          unicode LXGWWenKaiLite-Regular.ttf}
  \special{pdf:mapline gbksong@UGBK@      unicode LXGWWenKaiLite-Regular.ttf}
  \special{pdf:mapline gbkkai@UGBK@       unicode LXGWYozai-Regular.ttf}
  \special{pdf:mapline gbkhei@UGBK@       unicode LXGWMarkerGothic-Regular.ttf}
  \special{pdf:mapline gbkfs@UGBK@        unicode LXGWXiaolai-Regular.ttf}
  \special{pdf:mapline cyberb@Unicode@    unicode LXGWWenKaiLite-Regular.ttf}
  \special{pdf:mapline unisong@Unicode@   unicode LXGWWenKaiLite-Regular.ttf}
  \special{pdf:mapline unikai@Unicode@    unicode LXGWYozai-Regular.ttf}
  \special{pdf:mapline unihei@Unicode@    unicode LXGWMarkerGothic-Regular.ttf}
  \special{pdf:mapline unifs@Unicode@     unicode LXGWXiaolai-Regular.ttf}
  \special{pdf:mapline gbksongsl@UGBK@    unicode LXGWWenKaiLite-Regular.ttf -s .167}
  \special{pdf:mapline gbkkaisl@UGBK@     unicode LXGWYozai-Regular.ttf -s .167}
  \special{pdf:mapline gbkheisl@UGBK@     unicode LXGWMarkerGothic-Regular.ttf -s .167}
  \special{pdf:mapline gbkfssl@UGBK@      unicode LXGWXiaolai-Regular.ttf -s .167}
  \special{pdf:mapline unisongsl@Unicode@ unicode LXGWWenKaiLite-Regular.ttf -s .167}
  \special{pdf:mapline unikaisl@Unicode@  unicode LXGWYozai-Regular.ttf -s .167}
  \special{pdf:mapline uniheisl@Unicode@  unicode LXGWMarkerGothic-Regular.ttf -s .167}
  \special{pdf:mapline unifssl@Unicode@   unicode LXGWXiaolai-Regular.ttf -s .167}
\fi
%    \end{macrocode}
% End the optionlist |zhmap| for l3docstrip.
%    \begin{macrocode}
%<*zhmap>
%    \end{macrocode}
% \end{implementation}